% 格式:report
\documentclass[12pt,a4paper]{report}

% 中文字形
\usepackage{CJK}

% 行距設定
\renewcommand{\baselinestretch}{1.5} 

% 頁首頁尾距離
% Flexible and complete interface to document dimensions
% https://ctan.org/pkg/geometry
\usepackage{geometry}
\geometry{a4paper,left=2.54cm,right=2.54cm,top=2.54cm,bottom=2.54cm}

\title{Real Analysis}

\author{}
\date{}

\begin{document}
\begin{CJK}{UTF8}{bsmi}%\begin{CJK}{UTF8}{bsmi}
\CJKindent

\maketitle

\begin{abstract}


司法差距是各國政府長期困擾的社會議題,臺灣法律扶助基金會為國內針對此議題之對策組織,礙於組織不諳統計方法,難以量化服務需求找出潛在之重點服務區。本研究期望以空間統計方法初探法律扶助案件之分布,利用區域資料分析,探討法律扶助案件數於北北基各經濟二級發布區之分布樣態,建構合適的貝氏階層線性模型,以標示出法律扶助案件申請之潛在需求區,並推論影響案件數的因子及其效果。首先,利用 Moran's I 指標發現案件數分布具空間自我相關性,再以 CAR 模型設定研究區域之空間結構,將隨機效果項納入廣義線性模型,使用 R 語言 INLA 套件進行參數之貝氏估計。最後,本研究比較 Besag-York-Mollie、Leroux Model 及 Locally Adaptive Model 3~種 CAR 模型之空間結構設定,以 Besag-York-Mollie 模型為最適解釋法律扶助案件申請的分析模型,供法扶基金會作為後續擬策參考。

\end{abstract}

\end{CJK}
\end{document}